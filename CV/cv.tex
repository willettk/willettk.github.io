% Date of last change: 7/12/11

\documentstyle[11pt,fancyhdr]{article}
\marginparwidth 90pt


\marginparsep 11pt
\topmargin -27pt\headheight12pt \headsep 25pt \footheight 12pt
\textheight 9.0in \columnsep10pt 
\oddsidemargin 0pt\evensidemargin0pt
\raggedright\raggedbottom
\widowpenalty=10000

\textwidth6.5in

\begin{document}
\pagestyle{fancy}
\fancyhead[L]{}
\fancyhead[R]{}
\renewcommand{\headrulewidth}{1pt} %Thin line separating header from body  
%\fancyfoot{\small\begin{center}\thepage\ of \pageref{maxpg}\end{center}}
\parindent0pt

\begin{center}
{\LARGE\bf Kyle W. Willett}\\
\medskip\medskip
\begin{tabular}{lcr}
School of Physics and Astronomy  &  \hspace{70pt} &  (612) 626-0405 \\
University of Minnesota       &  \hspace{70pt} &  {\tt willett@physics.umn.edu}       \\
Minneapolis, MN 55455     &  \hspace{70pt} &  http://homepages.spa.umn.edu/$\sim$willett/ \\
\end{tabular}
\end{center}

\medskip
\noindent{\Large\bf Education}
\vskip4pt
\medskip

\begin{tabular}{@{}p{3.3in}p{2.8in}}
{\large \sc Ph.D., Astrophysical and Planetary}     & {\large\bf University of Colorado}, Boulder, CO \\
{\large \sc Sciences, 2011}                         & {\it ``OH Masers from Andromeda to the Peak of Cosmic Star Formation''} \\[4pt]
                                          & Thesis advisor: Prof. Jeremy Darling\\[4pt]
{\large \sc M.S., Astrophysical, Planetary, and }     & {\large\bf University of Colorado}, Boulder, CO\\
{\large \sc Atmospheric Sciences, 2007}   &                                                \\[4pt]
{\large \sc B.A., Physics, 2005}          & {\large\bf Carleton College}, Northfield, MN\\  
{\small {\it (magna cum laude)}}          & \\
\end{tabular}

%\medskip
%\medskip
%\noindent{\Large\bf Objectives}

\medskip
\noindent{\Large\bf Research Interests}
\vskip2pt
\medskip
%\begin{tabular}{@{}p{1em}p{5.in}}
\begin{tabular}{llll}
\hspace{50pt} & {$\bullet$ \large Galaxy morphology}  & \hspace{50pt} & {$\bullet$ \large Galaxy evolution}             \\
\hspace{50pt} & {$\bullet$ \large Blazars}        & \hspace{50pt} & {$\bullet$ \large Citizen science} \\
\hspace{50pt} & {$\bullet$ \large Astrophysical masers }   & \hspace{50pt} & {$\bullet$ \large Active galactic nuclei}          \\
\end{tabular}

\medskip\medskip
\noindent{\Large\bf Research Experience}\\
\medskip

\hangindent2.2em{\large {\bf{\sc Research Associate}, University of Minnesota} (2011--present)}.  Postdoctoral research with the group of Prof. Lucy Fortson. Lead data scientist for the Galaxy Zoo project (2015--). Led reduction and analysis of the Galaxy Zoo 2 project. Worked on environmental properties of blazars/BL Lacs and the evolutionary blazar sequence. Developed new research tools for citizen science projects at the Zooniverse. \\[4pt]
\hangindent2.2em{\large {\bf{\sc Graduate Research Assistant}, University of Colorado} (2006--2011)}.  Ph.D. thesis research on OH masers, including both single-dish and interferometric searches for masers as well as ground- and space-based infrared studies of maser host galaxies with Prof.~Jeremy Darling. Additional research on infrared properties of young radio galaxies with Prof. John Stocke.\\[4pt]

\hangindent2.2em{\large {\bf{\sc Junior Research Associate}, National Radio Astronomy Observatory} (Jun--Aug~2006)}. Reduction and analysis of a high-frequency radio survey for molecular absorbers at redshift $z\sim1$, with Drs. Chris Carilli and Nissim Kanekar. \\[4pt]

\medskip
\medskip
\noindent{\Large\bf Teaching Experience}\\[4pt]
\parindent 0.0em
\hangindent2.5em{\large {\bf{\sc Lecturer in Physics}, University of Minnesota} (Jan--May 2014). Instructor for PHYS 1101 course, ``Introductory College Physics I'' (classical mechanics for non-major undergraduates). Created and supervised lectures, assignments, exams, and lab activities for a 200-student course.\\[4pt]

\parindent 0.0em
\hangindent2.5em{\large {\bf{\sc Lecturer in Astronomy}, University of Colorado} (June--July 2009). Instructor of record for the ASTR 1110 course ``The Solar System''. Created all lectures, assignments, exams, and observing activities for a 30-student course.\\[4pt]

%\parindent 0.0em
%\hangindent2.5em{\large {\bf{\sc Grader}, University of Colorado} (2007--2009)}. Grader for newly-developed ``Observations, Data Analysis, \& Statistics'' core class for astrophysics graduate students. Assisted in modification of homework assignments and student-led analysis projects. \\[4pt]
%
\parindent 0.0em
\hangindent2.5em{\large {\bf{\sc Teaching Assistant}, University of Colorado} (2005--2006)}. Instructor for two laboratory sections of ``Introductory Astronomy'' and for two laboratory sections of ``Accelerated Introductory Astronomy''. Designed a new lab activity for the accelerated course and rewrote the solution set for the coursewide lab manual. \\[4pt]

\parindent 0.0em
\hangindent2.5em{\large {\bf{\sc Graduate Teacher Program}, University of Colorado} (2005--2011). Participated in over 30 hours of educational workshops and seminars through the Graduate Teacher Program, including both astronomy-specific and general pedagogy. Included faculty observation and feedback, video recording, and one-on-one evaluations of my teaching performance. \\[4pt]

%\newpage

\pagestyle{fancy}
\fancyhead[L]{Kyle Willett}
\fancyhead[R]{Curriculum vit\ae}
\renewcommand{\headrulewidth}{1pt} %Thin line separating header from body  
%\fancyfoot{\small\begin{center}\thepage\ of \pageref{maxpg}\end{center}}
\parindent0pt

{\Large\bf Observational Experience}\\
\medskip
\parindent .5em
$\bullet$ Principal investigator on Green Bank Telescope proposal 10B-035 (70~hrs.) \\[2pt] %({\it student support \$\$?})
$\bullet$ Principal investigator on Arecibo Observatory proposal A2505 (3~hrs.) \\[2pt]
$\bullet$ Co-investigator on Green Bank Telescope proposals 08A-043 (25~hrs.), 08B-035 (71~hrs.), 10C-039 (30~hrs.)\\[2pt]
$\bullet$ Co-investigator on VLA proposal AD583 (25~hrs.) \\[2pt]
$\bullet$ Co-investigator on Spitzer Space Telescope proposals 50591, 80070 \\[2pt]
$\bullet$ Near-infrared observing at Apache Point 3.5m ARC telescope (3 half-nights). \\[2pt]
$\bullet$ Two on-site observing runs for radio pulsars at Parkes Observatory, NSW, Australia. \\[2pt]

\medskip
\medskip
\noindent
{\Large\bf Awards and Recognition}\\[4pt]
\medskip
\parindent .5em
$\bullet$ Ph.D. candidacy exam {\bf high pass}, University of Colorado (2007)\\[2pt]
$\bullet$ {\bf Phi Beta Kappa}, Carleton College (2005)\\[2pt]
$\bullet$ {\bf Lawrence Gould Prize in Natural Science}, Carleton College (2005)\\[2pt]
$\bullet$ {\bf Mike Ewers Award in Astronomy}, Carleton College (2004)\\[2pt]

\parindent0in
\medskip

\medskip
\noindent
{\Large\bf Service and Membership}\\[4pt]
\medskip
\parindent .5em
$\bullet$ {\bf Referee} for {\it ApJ, MNRAS, PASP} (2010--present)\\[2pt]
$\bullet$ {\bf Full Member}, American Astronomical Society (2005--present)\\[2pt]
$\bullet$ Graduate student {\bf committee member}, Dept. of Astrophysics and Planetary Science, Univ. of Colorado: colloquium (2005--06), graduate exams (2007), graduate admissions (2008), faculty representative (2009--2010)\\[2pt]
$\bullet$ Sommers-Bausch Observatory {\bf open house volunteer} (2005--2011)\\[2pt]
$\bullet$ Astronomy Day {\bf volunteer}, University of Colorado (2005--2011)\\[2pt]

\medskip
\noindent
{\Large\bf Invited talks}\\[4pt]
\medskip
\parindent .5em
$\bullet$ {\bf University of Colorado}, March 2008 \\[2pt]
$\bullet$ {\bf Minnesota Astronomical Society}, August 2011 \\[2pt]
$\bullet$ {\bf Augsburg College}, October 2011 \\[2pt]
$\bullet$ {\bf University of Minnesota}, November 2011 \\[2pt]
$\bullet$ {\bf Yale University}, April 2013 \\[2pt]
$\bullet$ {\bf University of Portsmouth}, May 2013 \\[2pt]
$\bullet$ {\bf University of Oxford}, May 2013 \\[2pt]
$\bullet$ {\bf Royal Astronomical Society}, May 2013 \\[2pt]
$\bullet$ {\bf Minnesota State University, Mankato}, Feb 2014 \\[2pt]
$\bullet$ {\bf Bell Museum, University of Minnesota}, Feb 2015 \\[2pt]
$\bullet$ {\bf Grinnell College}, Apr 2015 \\[2pt]
$\bullet$ {\bf University of Kentucky}, Feb 2016 \\[2pt]

\parindent0in
\medskip
\medskip
\noindent
{\Large\bf Professional references}\\[4pt]
\medskip
\parindent .5em

\begin{tabular}{@{}p{3.5in}p{3.5in}}
{\large \bf Prof. Jeremy Darling}             &     {\large \bf Prof. Chris Lintott}        \\	       
University of Colorado                        &     University of Oxford                  \\      
Dept. of Astrophysics and Planetary Science   &     Dept. of Physics  	                  \\   
UCB 391		                                  &	    Denys Wilkinson Building, Keble Road  \\ 
Boulder, CO  80309                            &     Oxford OX1 3RH, UK                    \\
{\tt jdarling@colorado.edu}                   &     {\tt cjl@astro.ox.ac.uk}              \\
(303) 492-4881                                &     +44 01865 (2)73638                    \\[8pt]

{\large \bf Prof. Kevin Schawinski}           &     {\large \bf Prof. Lucy Fortson}       \\
ETH Z{\"u}rich                                &     University of Minnesota               \\
Dept. of Physics                              &     School of Physics \& Astronomy        \\   
Wolfgang-Pauli-Str. 27                        &     116 Church St. SE                     \\
8093 Zürich, Switzerland                      &     Minneapolis, MN  55455                \\        
{\tt kevin.schawinski@phys.ethz.ch}  	      &     {\tt fortson@physics.umn.edu}         \\        
+41 44 633 76 08                              &     (612) 624-9587                        \\        
                                                                
% {\large \bf Prof. Eric Perlman}      \\
% Dept. of Physics \& Astronomy	       \\   
% Florida Institute of Technology      \\
% 150 West University Boulevard        \\
% Melbourne, FL  32901                 \\        
% {\tt eperlman@fit.edu}               \\        
% (321) 674-7741                       \\        
                                                                
% {\large \bf Prof. Vassilis Charmandaris} \\	       
% University of Crete                      \\      
% Dept. of Physics  	                   \\   
% GR-71003                                 \\ 
% Heraklion, Greece                        \\
% {\tt vassilis@physics.uoc.gr}            \\
% +30 2810-394216                          \\

\end{tabular}

\label{maxpg}

\end{document}
